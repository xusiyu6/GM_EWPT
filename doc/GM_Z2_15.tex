\documentclass[12pt]{article}
\pdfoutput=1 % if your are submitting a pdflatex (i.e. if you have images in pdf, png or jpg format)

\usepackage{jheppub} % for details on the use of the package, please see the JHEP-author-manual
\usepackage{graphicx}  % needed for figures
\usepackage{dcolumn}   % needed for some tablese$\cdot$cm
\usepackage{bm}        % for math
\usepackage{amssymb}   % for math
\usepackage{subfigure}
\usepackage{epstopdf}
\usepackage{color}
\usepackage{slashed}
\usepackage[utf8]{inputenc}
\usepackage{multirow}
\usepackage{footnote}
\usepackage{amsmath}
\allowdisplaybreaks[4]
\usepackage{accents}
\newlength{\dhatheight}
\newcommand{\doublehat}[1]{%
    \settoheight{\dhatheight}{\ensuremath{\hat{#1}}}%
    \addtolength{\dhatheight}{-0.35ex}%
    \hat{\vphantom{\rule{1pt}{\dhatheight}}%
    \smash{\hat{#1}}}}
\newcommand{\ycwu}[1]{{\bf \color{red} yc: #1}}
\usepackage{hyperref}
% autoref configuration
\def\figureautorefname~#1\null{Fig.\,#1\null}
\def\tableautorefname~#1\null{Tab.\,#1\null}
\renewcommand{\sectionautorefname}{Section}
\renewcommand{\subsectionautorefname}{Section}
\def\equationautorefname~#1\null{Eq.\,(#1)\null}

\title{The Effective Potential for EWPT in $Z_2$ symmetric GM model with only $\omega_1$ and $\omega_5$ as background field}
\author[a]{ Yongcheng Wu}
\author[a]{ Siyu Xu}
\affiliation[a]{Department of Physics and Institute of Theoretical Physics, Nanjing Normal University, Nanjing, 210023, China}
% \affiliation[b]{Institute-b}
% \affiliation[c]{Institute-c}
% \affiliation[d]{Institute-d}
\emailAdd{ycwu@njnu.edu.cn}
\emailAdd{xxxxxx}


%\date{\today}

\abstract{
This is the document containing some details of the effective potential in the $Z_2$ symmetric GM model with only $\omega_1$ and $\omega_2$ as background field kept as a reference for the corresponding code.
}


\begin{document}
\titlepage
\maketitle
\newpage

\flushbottom

\section{The Model}

In the GM model, along with the standard scalar doublet, we have two extra triplets. In bi-doublet and bi-triplet form, these fields are given as
\begin{align}
\Phi &= \frac{1}{\sqrt{2}}\begin{pmatrix}
    \omega_1 + \phi_1 - i \phi_2 & \phi_3 + i \phi_4 \\
    -\phi_3 + i \phi_4 & \omega_1 + \phi_1 + i \phi_2
\end{pmatrix}\\
 X &= \frac{1}{\sqrt{2}}\begin{pmatrix}
    \omega_8+\phi_8-i\phi_9 & \phi_6+i\phi_7 & \phi_{12}+i\phi_{13}\\
    -\phi_{10}+i\phi_{11} & \sqrt{2}(\omega_5+\phi_5) & \phi_{10}+i\phi_{11}\\
    \phi_{12}-i\phi_{13}& -\phi_6+i\phi_7 & \omega_8 +\phi_8 +i\phi_9
\end{pmatrix}
\end{align}
where we only consider the custodial symmetric case with $\omega_8 = \sqrt{2}\omega_5$. $\phi_i$'s are all real scalar components, while $\omega_i$'s are background fields.

The tree-level potential with a $Z_2$ symmetry is given by
\begin{align}
    \label{equ:V0}
    V_0 &= \frac{\mu_2^2}{2}{\rm Tr}(\Phi^\dagger\Phi)+\frac{\mu_3^2}{2}{\rm Tr}(X^\dagger X) + \lambda_1\left({\rm Tr}(\Phi^\dagger\Phi)\right)^2 + \lambda_2{\rm Tr}(\Phi^\dagger\Phi){\rm Tr}(X^\dagger X) \nonumber \\
    & \qquad + \lambda_3{\rm Tr}(X^\dagger X X^\dagger X) + \lambda_4\left({\rm Tr}(X^\dagger X)\right)^2 - \lambda_5{\rm Tr}(\Phi^\dagger \tau^a\Phi\tau^b){\rm Tr}(X^\dagger t^a X t^b)
\end{align}
where $\tau^a=\sigma^a/2$ with $\sigma^a$ being the Pauli matrices. The generators for the triplet representation are given as
\begin{align}
    t^1 = \frac{1}{\sqrt{2}}\begin{pmatrix}
        0 & 1 & 0 \\
        1 & 0 & 1 \\
        0 & 1 & 0
    \end{pmatrix}, \quad t^2=\frac{1}{\sqrt{2}}\begin{pmatrix}
        0 & -i & 0 \\
        i & 0 & -i \\
        0 & i & 0
    \end{pmatrix}, \quad t^3=\begin{pmatrix}
        1 & 0 & 0 \\
        0 & 0 & 0 \\
        0 & 0 & -1
    \end{pmatrix}.
\end{align}


\section{The effective potential}

\subsection{The tree-level effective potential}

After expansion and keeping the terms only involving $\omega_1$ and $\omega_5$, we have
\begin{align}
    \label{equ:V0eff}
    V_0 &= \frac{\mu_2^2}{2}\omega_1^2 + \frac{3\mu_3^2}{2}\omega_5^2 + \lambda_1 \omega_1^4 + 3\lambda_2 \omega_1^2 \omega_5^2 +3\lambda_3\omega_5^4 + 9\lambda_4\omega_5^4-\frac{3\lambda_5}{2}\omega_1^2\omega_5^2
\end{align}

\subsubsection{Zero temperature relations}

At zero temperature, the theory will provide one scalar and three goldstone bosons. Further, the theory should provide correct vev at zero temperature to give the correct (measured) mass of the gauge boson and fermions. We only consider the tree-level relations for these considerations.

The vev at zero temperature is determined by the minimum position of~\autoref{equ:V0}, while the mass of the scalar is determined by the quadruple terms in~\autoref{equ:V0}:
\begin{align}
    \label{equ:stationary_condition}
    \begin{cases}
    \left.\frac{\partial V_0}{\partial \omega_1}\right|_{\phi_i=0,\omega_1 = v_1,\omega_5=v_5} = 0 \\
    \left.\frac{\partial V_0}{\partial \omega_5}\right|_{\phi_i=0,\omega_1 = v_1,\omega_5=v5} = 0
    \end{cases} \Rightarrow \begin{cases}
    \mu_2^2 = -4\lambda_1v_1^2+3(\lambda_5-2\lambda_2)v_5^2\\
    \mu_3^2 = (\lambda_5-2\lambda_2)v_1^2-4(\lambda_3+3\lambda_4)v_5^2
    \end{cases}
\end{align}
where $v_{1,5}$ is the vev at zero temperature.

The mass of the scalars are determined by the quadratic terms of the potential. After electroweak symmetry breaking, in current case, we preserve the custodial symmetry (at zero temperature, it is better to keep the custodial symmetry). The scalars are grouped into different multiplets under the custodial symmetry:
\begin{subequations}
    \begin{align}
        H_5^{\pm\pm} &= \frac{\phi_{12}\pm i\phi_{13}}{\sqrt{2}}\\
        H_5^\pm &= \frac{(\phi_{10}-\phi_6) \pm i(\phi_{11}-\phi_7)}{2}\\
        H_5^0 &= \sqrt{\frac{2}{3}}\phi_5 - \sqrt{\frac{1}{3}}\phi_8\\
        H_3^\pm &= - \frac{s_H}{\sqrt{2}}(\phi_3\pm i\phi_4) + \frac{c_H}{2}((\phi_{10}+\phi_6)\pm i(\phi_{11}+\phi_7))\\
        H_3^0 &= -s_H \phi_2 + c_H \phi_9\\
        G^\pm &= \frac{c_H}{\sqrt{2}}(\phi_3\pm i\phi_4)+\frac{s_H}{2}((\phi_{10}+\phi_6)\pm(\phi_{11}+\phi_7))\\
        G^0 &= c_H \phi_2 + s_H \phi_9\\
        H_1^0 &= \phi_1 \\
        H_1^{0\prime} &= \sqrt{\frac{1}{3}}\phi_5 + \sqrt{\frac{2}{3}}\phi_8
    \end{align}
\end{subequations}
The masses for the fiveplet and triplets are
\begin{align}
    m_5^2 &= \frac{3}{2}\lambda_5 v_1^2 + 8\lambda_3 v_5^2\\
    m_3^2 &= \frac{1}{2}\lambda_5 (v_1^2 + 8 v_5^2)
\end{align}
The singlets listed above will further mix to form the mass eigenstates:
\begin{align}
\label{equ:mixing_singlet}
\begin{pmatrix}
    h\\
    H
\end{pmatrix} = \begin{pmatrix}
    c_\alpha & -s_\alpha \\
    s_\alpha & c_\alpha
\end{pmatrix}\begin{pmatrix}
    H_1^0\\
    H_1^{0\prime}
\end{pmatrix}
\end{align}
The mass matrix in the $(H_1^0, H_1^{0\prime})$ basis is
\begin{align}
\label{equ:mass_singlet}
\mathcal{M}^2 = \begin{pmatrix}
    8\lambda_1v_1^2 & 2\sqrt{3}(2\lambda_2 - \lambda_5)v_1v_5\\
    2\sqrt{3}(2\lambda_2-\lambda_5)v_1v_5 & 8(\lambda_3+3\lambda_4)v_5^2
\end{pmatrix}
\end{align}
On the other hand, $(h,H)$ are the mass eigenstates, hence the corresponding mass matrix is
\begin{align}
\label{equ:mass_singlet_diag}
\mathcal{M}^2_E = \begin{pmatrix}
    m_h^2 & 0 \\
    0 & m_H^2
\end{pmatrix}
\end{align}
The mass matrix in $(h,H)$ basis~\autoref{equ:mass_singlet_diag} and that in $(H_1^0,H_1^{0\prime})$~\autoref{equ:mass_singlet} are related by the mixing matrix in~\autoref{equ:mixing_singlet}:
\begin{align}
    R^T(\alpha)\mathcal{M}_E^2R(\alpha) &= \mathcal{M}^2\nonumber\\
    \begin{pmatrix}
        c_\alpha & s_\alpha \\
        -s_\alpha & c_\alpha
    \end{pmatrix}\begin{pmatrix}
        m_h^2 & 0 \\
        0 & m_H^2
    \end{pmatrix}\begin{pmatrix}
        c_\alpha & -s_\alpha \\
        s_\alpha & c_\alpha
    \end{pmatrix} &= \begin{pmatrix}
        8\lambda_1v_1^2 & 2\sqrt{3}(2\lambda_2 - \lambda_5)v_1v_5\\
        2\sqrt{3}(2\lambda_2-\lambda_5)v_1v_5 & 8(\lambda_3+3\lambda_4)v_5^2
    \end{pmatrix}\nonumber\\
    \begin{pmatrix}
        m_h^2c_\alpha^2+m_H^2s_\alpha^2 & s_\alpha c_\alpha (m_H^2-m_h^2)\\
        s_\alpha c_\alpha (m_H^2-m_h^2) & m_h^2 s_\alpha^2 + m_H^2 c_\alpha^2
    \end{pmatrix} &= \begin{pmatrix}
        8\lambda_1v_1^2 & 2\sqrt{3}(2\lambda_2 - \lambda_5)v_1v_5\\
        2\sqrt{3}(2\lambda_2-\lambda_5)v_1v_5 & 8(\lambda_3+3\lambda_4)v_5^2
    \end{pmatrix}
\end{align}
We would like to use the zero-temperature masses as input instead of $\lambda_i$'s. Hence we have the following five relationships:
\begin{subequations}
    \begin{align}
        m_5^2 &= \frac{3}{2}\lambda_5v_1^2 + 8\lambda_3 v_5^2\\
        m_3^2 &= \frac{1}{2}\lambda_5(v_1^2 + 8v_5^2) \\
        m_h^2c_\alpha^2 + m_H^2 s_\alpha^2 &= 8\lambda_1v_1^2 \\
        m_h^2s_\alpha^2 + m_H^2 c_\alpha^2 &= 8(\lambda_3+3\lambda_4)v_5^2\\
        s_\alpha c_\alpha (m_H^2 - m_h^2) &= 2\sqrt{3}(2\lambda_2-\lambda_5)v_1 v_5
    \end{align}
\end{subequations}
Then we have
\begin{subequations}
    \begin{align}
        \lambda_1 &= \frac{m_h^2 c_\alpha^2 + m_H^2 s_\alpha^2}{8v_1^2}\\
        \lambda_2 &= \frac{m_3^2}{v_1^2 + 8 v_5^2} + \frac{s_{2\alpha}}{8\sqrt{3}}\frac{m_H^2-m_h^2}{v_1v_5}\\
        \lambda_3 &= \frac{m_5^2}{8v_5^2} - \frac{3v_1^2}{8v_5^2}\frac{m_3^2}{v_1^2+8v_5^2}\\
        \lambda_4 &= \frac{v_1^2}{8v_5^2}\frac{m_3^2}{v_1^2+8v_5^2}+\frac{m_h^2s_\alpha^2+m_H^2c_\alpha^2-m_5^2}{24v_5^2}\\
        \lambda_5 &= \frac{2m_3^2}{v_1^2+8v_5^2}
    \end{align}
\end{subequations}
Together with~\autoref{equ:stationary_condition}, we can treat $\mu_2^2,\mu_3^2,\lambda_{1,\cdots,5}$ as functions of $(v_1,v_5,\alpha,m_h^2,m_H^2,m_3^2,m_5^2)$ (the zero-temperature physical quantities).

Further, we will also need to provide the correct mass for gauge bosons which is obtained from the kinematic terms of the scalars:
\begin{align}
\mathcal{L}_K = \frac{1}{2}{\rm Tr}\left((D_\mu\Phi)^\dagger(D^\mu\Phi)\right) + \frac{1}{2}{\rm Tr}\left((D_\mu X)^\dagger (D^\mu X)\right)
\end{align}
At the stationary point (and zero temperature), we have the mass matrix for the gauge bosons in basis $(W_1,W_2,W_3,B)$
\begin{align}
    \mathcal{M}_G^2 = \frac{1}{4}(v_1^2+8v_5^2)\begin{pmatrix}
        g^2 & 0 & 0 & 0 \\
        0 & g^2 & 0 & 0 \\
        0 & 0 & g^2 & -g^2t_{\theta_w} \\
        0 & 0 & -g^2t_{\theta_w} & g^2t_{\theta_w}
    \end{pmatrix},
\end{align}
which provides a constraint at zero temperature as
\begin{align}
    v_1^2 + 8v_5^2 = v^2 = (246\,{\rm GeV})^2
\end{align}
Hence, we would like to introduce another parameter $\theta_H$ to parameterize the relation between $v_1$ and $v_5$:
\begin{align}
    \cos\theta_H \equiv \frac{v_1}{v},\qquad  \sin\theta_h \equiv \frac{2\sqrt{2}v_5}{v}
\end{align}
Then finally, we reach the full parameter set (at zero temperature) that will be used as input parameters for the model. All other parameters (either at zero temperature or at finite temperature) will be expressed in terms of this parameter set:
\begin{align}
    v(=246\,{\rm GeV}), s_H(\equiv \sin\theta_H), \alpha, m_h(=125\,{\rm GeV}), m_H, m_3, m_5
\end{align}

On the other hand, the vevs of the scalars also provide the mass for the fermions. In GM model, only the doublet can provide the mass for the fermions in the same way as in SM. Note that we ignore the contribution from the triplet for the neutrinos:
\begin{align}
    \mathcal{L}_Y = -y_u^i \bar{Q}_L^i\tilde{\phi}u_R - y_d^i \bar{Q}_L^i\phi d_R + h.c.
\end{align}
where, $\phi = (\phi_3+i\phi_4, \omega_1+\phi_1+i\phi_2)^T/\sqrt{2}$ is the doublet. In practice, we only consider the 3rd generation quarks, then we have
\begin{align}
    \mathcal{L}_Y^{3q} = -y_t \bar{Q}_L^3\tilde{\phi}t_R - y_b \bar{Q}_L^3 \phi b_R +h.c.
\end{align}
where $Q_L^3 = (t_L, b_L)^T$. The above Yukawa couplings provide the mass for the 3rd generation quarks after the Electroweak symmetry breaking at zero-temperature:
\begin{align}
    m_t = \frac{y_tv_1}{\sqrt{2}},\quad m_b = \frac{y_bv_1}{\sqrt{2}}.
\end{align}

\subsection{One loop correction at zero temperature}

Here, we neglect the derivation of the Coleman-Weinberg potential and just list the result, and we only consider the case in Landau gauge and in $\overline{\text{MS}}$ scheme:

\begin{align}
    V_{1,T=0}^{\overline{\text{MS}},\text{Landau}} &= \frac{1}{64\pi^2}\sum_i(-1)^{2s_i}(1+2s_i)m_i^4(\omega)\left(\log\left(\frac{m_i^2(\omega)}{\mu^2}\right)-k_i\right) \nonumber\\
    &=\frac{1}{64\pi^2}\sum_f(-1)^{2s_F}n_Fm_F^4(\omega)\left(\log\left(\frac{m_F^2(\omega)}{\mu^2}\right)-k_F\right)
\end{align}
where $i$ runs over all real scalar, Weyl fermion and vector boson degrees of freedom, (hence the summation should be run over any internal d.o.f like colors), $s_i$ is the spin of the field $i$, and $k_i = 3/2$ for scalars and fermions and $5/6$ for gauge bosons, and $m_i^2(\omega)$ is the squared field dependent mass. In the second line, we combined the summation for the internal d.o.f, with the $n_F$ listed in the second column of~\autoref{tab:fields}.

In $Z_2$ symmetric GM model, the fields we need to sum over are given in~\autoref{tab:fields}. For fermions, we only consider the 3rd generation quarks.

\begin{table}
    \centering
    \resizebox{\textwidth}{!}{
    \begin{tabular}{|c|c|c|c|}
        \hline
        Fields & $n_F$ & $m_F^2(\omega)$ \\
        \hline
        $h$ & 1 & $\frac{\mu_2^2 + \mu_3^2 + (12\lambda_1+\lambda_{25})\omega_1^2 + 3(\lambda_{25}+4\lambda_{34})\omega_5^2 - \left((\mu_2^2-\mu_3^2+(12\lambda_1-\lambda_{25})\omega_1^2)^2+9(4\lambda_{34}-\lambda_{25})^2\omega_5^4+6((12\lambda_1(\lambda_{25}-4\lambda_{34})+\lambda_{25}(7\lambda_{25}+4\lambda_{34}))\omega_1^2+(\lambda_{25}-4\lambda_{34})(\mu_2^2-\mu_3^2))\omega_5^2\right)^{1/2} }{2}$ \\
        \hline
        $H$ & 1 & $\frac{\mu_2^2 + \mu_3^2 + (12\lambda_1+\lambda_{25})\omega_1^2 + 3(\lambda_{25}+4\lambda_{34})\omega_5^2 + \left((\mu_2^2-\mu_3^2+(12\lambda_1-\lambda_{25})\omega_1^2)^2+9(4\lambda_{34}-\lambda_{25})^2\omega_5^4+6((12\lambda_1(\lambda_{25}-4\lambda_{34})+\lambda_{25}(7\lambda_{25}+4\lambda_{34}))\omega_1^2+(\lambda_{25}-4\lambda_{34})(\mu_2^2-\mu_3^2))\omega_5^2\right)^{1/2} }{2}$ \\
        \hline
        $H_3$ & 3 & $\frac{2(\mu_2^2+\mu_3^2)+(8\lambda_1+\lambda_{425})\omega_1^2+2(\lambda_{625}+4\lambda_{34})\omega_5^2+\left((2(\mu_2^2-\mu_3^2)+(8\lambda_1-\lambda_{425})\omega_1^2)^2+4(\lambda_{625}-4\lambda_{34})^2\omega_5^4+4(2(\lambda_{625}-4\lambda_{34})(\mu_2^2-\mu_3^2)+(2\lambda_5(4\lambda_1+\lambda_2-2\lambda_{34})+8(2\lambda_1-\lambda_2)(3\lambda_2-2\lambda_{34})+9\lambda_5^2)\omega_1^2)\omega_5^2\right)^{1/2}}{4}$ \\
        \hline
        $H_5$ & 5 & $\mu_3^2+\frac{1}{2}(4\lambda_2+\lambda_5)\omega_1^2+12(\lambda_3+\lambda_4)\omega_5^2$ \\
        \hline
        $G$ & 3 &  $\frac{2(\mu_2^2+\mu_3^2)+(8\lambda_1+\lambda_{425})\omega_1^2+2(\lambda_{625}+4\lambda_{34})\omega_5^2-\left((2(\mu_2^2-\mu_3^2)+(8\lambda_1-\lambda_{425})\omega_1^2)^2+4(\lambda_{625}-4\lambda_{34})^2\omega_5^4+4(2(\lambda_{625}-4\lambda_{34})(\mu_2^2-\mu_3^2)+(2\lambda_5(4\lambda_1+\lambda_2-2\lambda_{34})+8(2\lambda_1-\lambda_2)(3\lambda_2-2\lambda_{34})+9\lambda_5^2)\omega_1^2)\omega_5^2\right)^{1/2}}{4}$ \\
        \hline
        $W$ & 6 & $\frac{1}{4}g^2(\omega_1^2+8\omega_5^2)$ \\
        \hline
        $Z$ & 3 & $\frac{1}{4}\frac{g^2}{c_W^2}(\omega_1^2+8\omega_5^2)$ \\
        \hline
        $\gamma$ & 2 & $0$ \\
        \hline
        $t_L$ & 6 & $\frac{1}{2}y_t^2\omega_1^2$ \\
        \hline
        $\bar{t}_R$ & 6 & $\frac{1}{2}y_t^2\omega_1^2$ \\
        \hline
        $b_L$ & 6 & $\frac{1}{2}y_b^2\omega_1^2$ \\
        \hline
        $\bar{b}_R$ & 6 & $\frac{1}{2}y_b^2\omega_1^2$ \\
        \hline
    \end{tabular}}
    \caption{\label{tab:fields}The fields and corresponding d.o.f (including $(1+2s)$ factor) and field dependent mass in $Z_2$ symmetric GM model, where $\lambda_{25}=2\lambda_2 -\lambda_5$, $\lambda_{425}=4\lambda_2-\lambda_5$, $\lambda_{625}=6\lambda_2+\lambda_5$ and $\lambda_{34}=\lambda_3+3\lambda_4$. Note that the fermion mass should be understood as the eigenvalues of $M_F^\dagger M_F$.}
\end{table}

\subsection{The finite temperature corrections}

The thermal corrections to the effective potential in the Landau gauge read:
\begin{align}
    V_{1T}^{\text{Landau}} = \frac{T^4}{2\pi^2}\left(\sum_{b\in\text{Bosons}}n_b J_B\left(\frac{m_b^2(\omega)}{T^2}\right)-\sum_{f\in\text{Fermions}}n_fJ_F\left(\frac{m_f^2(\omega)}{T^2}\right)\right)
\end{align}
where the $n_i$ and the filed dependent masses are listed in~\autoref{tab:fields}.

\subsection{The Daisy resummations}

% \begin{acknowledgments}
% Acknowledgments
% \end{acknowledgments}

\bibliographystyle{bibsty}
\bibliography{references}

\end{document}
